%%=============================================================================
%% Methodologie
%%=============================================================================

\chapter{Methodologie}
\label{ch:methodologie}

%% TODO: Hoe ben je te werk gegaan? Verdeel je onderzoek in grote fasen, en
%% licht in elke fase toe welke stappen je gevolgd hebt. Verantwoord waarom je
%% op deze manier te werk gegaan bent. Je moet kunnen aantonen dat je de best
%% mogelijke manier toegepast hebt om een antwoord te vinden op de
%% onderzoeksvraag.

%%\lipsum[21-25]

In dit hoofdstuk zal er besproken worden welke methodes en manieren van werking er gehandhaafd werd voor uitschrijven van deze proef. Vervolgens zal er ook een duidelijke overzicht getoond worden wat men juist van elk hoofdstuk kan verwachten.

\section{Gehanteerde methodiek}

Het onderzoek is onderverdeeld in verschillende delen. In het eerste deel zal er eerder aangetoond worden wat IPv6 inhoud en hoe deze verschilt met zijn voorganger IPv4. Dit zal al een duidelijk overzicht moeten geven waarom er de noodzaak was om de overschakeling uit te voeren. Verder zijn er ook enkele transitie technieken die kunnen toegepast worden in een netwerk om mee te gaan in deze overgang. In het tweede grote deel van de proef zal er eerder een onderzoek gedaan worden naar de adoptie van IPv6 en hoe onze Belgische markt en bedrijven hierop anticiperen. Hoe IPv4 er momenteel voor staat en hoelang dit protocol nog zal overleven tot er niets anders meer is dan IPv6. Ook werden enkele vragen opgesteld en beantwoord door bedrijven over hun standpunt en visie over IPv6. 

Om deze scriptie tot een geslaagd succes te brengen, heb ik gekozen om te onderzoeken hoe Belgische bedrijven en ISP’s reageren op IPv6. Aan bedrijven zijn er enkele vragen opgesteld en beantwoord die een duidelijk overzicht geven over hoe zij erover denken, wat hun standpunt is, of er mogelijkse plannen zijn voor een uitbreiding of ze er momenteel mee aan het experimenteren zijn. Bij ISP’s was het belangrijk of hun klanten al begeleid werden met IPv6 en of hun apparatuur IPv6 ondersteunend zijn.

