%---------- Inleiding ---------------------------------------------------------

\section{Introductie} % The \section*{} command stops section numbering
\label{sec:introductie}

Mijn onderzoek zal gaan naar hoe de overschakeling binnen bedrijven van IPv4 (Internet Protocol version 4) naar IPv6 (Internet Protocol version 6) verloopt. Zijn de bedrijven in hun netwerk bezig naar het overschakelen naar het vernieuwde internet protocol, die noodzakelijk was binnen het internet, of ze het van plan of hebben ze er helemaal nog niet bij stilgestaan. Om dit onderzoek tot een succes uit te voeren, zal ik gebruik maken van een enquête en de antwoorden doorgronden of deze al dan niet terechte opmerkingen zijn.

%---------- Stand van zaken ---------------------------------------------------

\section{State-of-the-art}
\label{sec:state-of-the-art}

Deze scriptie zal gaan over de huidige stand van zaken van de adoptie van IPv6 op zowel globaal niveau als op Belgisch niveau. Er zal een studie uitgevoerd worden en enquêtes onderzocht worden. Dit dient ervoor om het standpunt van een bedrijf naar boven te brengen en de verkregen resultaten te doorgronden. 

% Voor literatuurverwijzingen zijn er twee belangrijke commando's:
% \autocite{KEY} => (Auteur, jaartal) Gebruik dit als de naam van de auteur
%   geen onderdeel is van de zin.
% \textcite{KEY} => Auteur (jaartal)  Gebruik dit als de auteursnaam wel een
%   functie heeft in de zin (bv. ``Uit onderzoek door Doll & Hill (1954) bleek
%   ...'')


%---------- Methodologie ------------------------------------------------------
\section{Methodologie}
\label{sec:methodologie}

Om mijn onderzoek uit te voeren zal ik gebruik maken van vragenlijsten. Hierdoor zullen mijn onderzoeksvragen beantwoord worden en is er een duidelijk overzicht over het gebeuren binnen een netwerk van een bedrijf. Aan de hand van deze antwoorden kan er gekeken worden of deze al dan niet doorgrond zijn en of deze terechte antwoorden zijn. Ook zullen er veel data van enquêtes overlopen worden en geanalyseerd. Deze zullen steeds grafisch voorgesteld worden in grafieken. 

%---------- Verwachte resultaten ----------------------------------------------
\section{Verwachte resultaten}
\label{sec:verwachte_resultaten}

De verwachte resultaten zijn dat men ziet dat enkele bedrijven al rekening houden met de overstapping naar het vernieuwde internet protocol IPv6. Ook zal het aantal niet immens groot zijn en zeker onder de helft van het aantal ondervragen bedrijven zitten. Alsook dat nog niemand de grootste stap richting de overgang heeft gemaakt en zeker niet volledig het IPv6 protocol zal hanteren maar eerder samen met IPv4.

%---------- Verwachte conclusies ----------------------------------------------
\section{Verwachte conclusies}
\label{sec:verwachte_conclusies}

Mijn verwachtingen zijn dus dat er van geen enkel van de ondervraagde bedrijven een bedrijf is waarin ze volledig IPv6 zullen hanteren. Ik vermoed ook dat uit het onderzoek zal blijken dat er toch enkele bedrijven er zich van bewust zijn en toch de stap aan het maken zijn voor eerder een gemengde oplossing te gebruiken, dus IPv4 en IPv6 samen. Maar het grootste deel zal zeker nog niet in de richting gaan van een mogelijke implementatie van het protocol.

