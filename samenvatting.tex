%%=============================================================================
%% Samenvatting
%%=============================================================================

% TODO: De "abstract" of samenvatting is een kernachtige (~ 1 blz. voor een
% thesis) synthese van het document.
%
% Deze aspecten moeten zeker aan bod komen:
% - Context: waarom is dit werk belangrijk?
% - Nood: waarom moest dit onderzocht worden?
% - Taak: wat heb je precies gedaan?
% - Object: wat staat in dit document geschreven?
% - Resultaat: wat was het resultaat?
% - Conclusie: wat is/zijn de belangrijkste conclusie(s)?
% - Perspectief: blijven er nog vragen open die in de toekomst nog kunnen
%    onderzocht worden? Wat is een mogelijk vervolg voor jouw onderzoek?
%
% LET OP! Een samenvatting is GEEN voorwoord!

%%---------- Nederlandse samenvatting -----------------------------------------
%
% TODO: Als je je bachelorproef in het Engels schrijft, moet je eerst een
% Nederlandse samenvatting invoegen. Haal daarvoor onderstaande code uit
% commentaar.
% Wie zijn bachelorproef in het Nederlands schrijft, kan dit negeren, de inhoud
% wordt niet in het document ingevoegd.

\IfLanguageName{english}{%
\selectlanguage{dutch}
\chapter*{Samenvatting}
%%\lipsum[1-4]
\selectlanguage{english}
}{}

%%---------- Samenvatting -----------------------------------------------------
% De samenvatting in de hoofdtaal van het document

\chapter*{\IfLanguageName{dutch}{Samenvatting}{Abstract}}

%%\lipsum[1-4]

Dit onderzoek zal gaan over de evolutie en adoptie van IPv6 op de Belgische markt. Het nieuwe internet protocol en de opvolger van IPv4. In 2012 was al bekend gemaakt dat het einde van de beschikbare adressen er zat aan te komen en dat er hiervoor een opvolger nodig was. Daarom werd , alsook in 2012, dag van IPv6 uitgeroepen. Hiermee was de opvolger direct voorgesteld en bekroond als officiële vervanger van IPv4. Momenteel is het 6 jaar later en is er nog niet veel meer sprake geweest van IPv6 of een effective overgang naar IPv6. Daarom zal dit onderzoek zich verdiepen in de hedendaagse adoptie van IPv6.

Er zal onderzocht worden waarom IPv6 er is gekomen en wat de positieve punten zijn aan dit protocol. Wat de verschillen zijn tussen IPv4 en IPv6, specifiek gericht op de headers van beide protocollen. Verder in deze scriptie zal er onderzocht worden wat mogelijke tunneltechnieken zijn en hoe de communicatie tussen beide protcollen kan verlopen.

Om op de meeste onderzoeksvragen een antwoord te vinden, zal er aangetoond worden wat de situatie is van IPv6 en hoe de wereld, bedrijven en ISP's zich hierop aanpassen. Hoe de overschakeling al dan niet positief aan verlopen. Na het lezen van deze scriptie zal men een beter inzicht moeten hebben over de huidige situatie en waarom deze nog niet zo hoog scoort. Ook België zal nader onderzocht worden over de hoe de Belgische bedrijven zich aanpassen en wat hun ondervinden zijn en wat men kan doen om te adoptiegroei te vergroten.





