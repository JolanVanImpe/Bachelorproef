
\chapter{Inleiding tot IPv6}
\label{ch:h1}

In dit hoofdstuk wordt er een inleiding tot IPv6, de geschiedenis en ontstaan van IPv6 doornomen. Alsook zullen er nieuwe elementen besproken worden die gehandhaafd worden in IPv6 en de reden van bestaan en waarom we dit protocol nodig zullen hebben.

\section{Waarom IPv6}

IPv4 werd ontwikkeld in de vroege jaren ’70 voor het communiceren tussen onderzoekers en academici in Amerika. Op dat moment werd er geen rekening gehouden met enkele elementen zoals genoeg adressen, extra beveiliging en Quality of Service, QoS. Het protocol heeft meer dan 30 jaar overleefd en heeft een belangrijke rol gespeeld in het internetrevolutie. Maar zelfs de slimste systemen verjaren en raken uiteindelijke verouderd. Dit was zeker het geval met IPv4. Vandaag de dag staat alles verbonden met het internet en met elkaar. De enorme opkomst van webshops, IoT, mobiel apparatuur, sociale netwerken en zo voort zorgt voor meer geconnecteerde apparaten.  Deze opkomst zorgde voor het bereiken van de limiet van IPv4 en leidde tot een opvolger namelijk IPv6. 
IPv6 werd ontwikkeld door het experimenteren met IPv4. Hierdoor werd al snel opgemerkt dat er geen limieten meer mochten zijn maar meer flexibiliteit en schaalbaarheid. IPv6 zorgde daarom voor de grote groei van het internetgebruik, mobiliteit en extra beveiliging op vlak van end-to-end beveiliging \autocite{Hagen2014}.

\section{Geschiedenis van IPv6}

Internet Engineering Taks Force (IETF) begon in de jaren 1990 met het ontwikkelen van een opvolger voor IPv4. Al snel werd er een oplossing voor de gelimiteerde tekortkomingen gezocht en extra functionaliteit toegewezen. Daarom lanceerde IETF, Internet Protocol Next Generation (IPng) zone in de jaren 1993. Dit werd gebruikt voor het onderzoeken van voorstellen en aanbevelingen van verdere procedures \autocite{Hagen2014}.

Op de Toronto IETF meeting in 1994 werd er het nieuwe protocol namelijk IPv6 voorgesteld. De bestuurders richtte een Address Lifetime Expectation (ALE) werkgroep op om na te gaan of de geschatte levenstijd van IPv4 IETF genoeg tijd gaf om een oplossing te vinden met nieuwe functionaliteit, of er enkel tijd over was om het adres probleem aan te gaan. ALE had op basis van toenmalig beschikbare statistieken geschat dat de uitputting van IPv4 plaats zou vinden tussen het jaar 2005 en 2011. Later in 1994 werd IPv6 goedgekeurd door de Internet Engineering Steering Groep \autocite{Hagen2014}.

Eén van de grootste uitdagen maar ook opportuniteiten van IPv6 is de herontwikkeling van netwerken in de toekomst. Dit is waar bedrijven hun grootse aandacht aan zouden moeten vestigen bij het overstappen naar een IPv6 netwerk. Zodanig er geen oude concepten worden meegenomen naar een nieuw protocol. Daarom is het belangrijk om bij de integratie volledig de architectuur te herstructureren \autocite{Hagen2014}. 

\section{Nieuwigheden in IPv6}

Bij de vernieuwde apparaten zal IPv6 steeds beschikbaar zijn voor configuratie. Als dit bij oudere toestellen nog niet het geval is, dan kan dit vaak via software upgrades. Enkele nieuwigheden bij IPv6 is een uitgebreid adres lengte, autoconfiguratie, het formaat van de header is eenvoudiger opgesteld en verbeterde steun voor extra opties en extensies \autocite{Hagen2014}.

\subsection{Uitgebreid adres lengte}

Het adresformaat bevat 128 bits, dit rekent uit op meer dan 340 biljoen verschillende adressen. Dit wil ook zeggen dat er meer adressen beschikbaar zijn dan korrels zand op de aarde. Dit zou het tekort aan adressen voor eens en voor altijd moeten oplossen \autocite{Hagen2014}.

\subsection{Autoconfiguratie}

Een nieuwe functionaliteit is het Stateless Address Autoconfiguration (SLAAC) mechanisme. Dit mechanisme zal ervoor zorgen dat het connecteren efficiënter zal verlopen bij vooral mobiele apparaten zoals smartphones wanneer ze zich in een onbekend netwerk bevinden. Dit zou het werk van een netwerk engineer moeten vergemakkelijken \autocite{Hagen2014}. 

\subsection{Eenvoudiger headerformaat}

De header van een IPv6 pakket is vereenvoudigd. De lengte zal een vaste lengte hebben van 40 bytes, wat het verwerken hiervan veel sneller maakt dan ervoor. De 40 bytes is onderverdeeld in twee maal 16 bytes voor het bestemmingsadres en bronadres en nog 8 bytes voor algemene header informatie \autocite{Hagen2014}.

\subsection{Verbeterde steun voor extra opties en extensies}

IPv6 heeft de optie om extension headers toe te voegen. Deze worden enkel toegevoegd als ze nodig zijn. Dit zorgt er alweer voor om het pakket sneller te verwerken. Routingheaders, QoS en beveiliging zijn enkele headers die kunnen meegegeven worden \autocite{Hagen2014}.

\section{Hebben we IPv6 echt nodig?}

Om hierop kort te antwoorden, ja. IPv6 is een noodzaak geworden in de internetwereld. We zijn op een moment gekomen dat het aantal beschikbare IPv4 adressen steeds dichter bij de nul komt. Dit is echter niet te vermijden met de dagelijkse groei van het internet. Daarom is er de overschakeling nodig naar IPv6. IPv4 had een limiet van 4,3 miljard adressen, waardoor deze limiet al snel werd bereikt \autocite{Hagen2014}. 