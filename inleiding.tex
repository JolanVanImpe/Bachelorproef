%%=============================================================================
%% Inleiding
%%=============================================================================

\chapter{Inleiding}
\label{ch:inleiding}

Dankzij de groei van het internet, internetgebruikers, Internet Of Things en aantal mensen op aarde zijn er steeds meer apparaten verbonden met het internet. Dit zorgt voor de steeds verdere uitputting van IPv4 protocol. Het tekort aan beschikbare IPv4 adressen komt steeds dichterbij en de laatste /8 blok is vrijgegeven. Dit wil zeggen dat het einde van IPv4 nadert. De reeds gelanceerde opvolger, IPv6, zal de problemen van IPv4 moeten oplossen. Hierdoor is er een onderzoek nodig naar de werking en verschillen van IPv6. Verder zal er aangetoond worden wat de huidige stand van zaken is over de adoptie van IPv6 en hoever deze al staat.

\section{Probleemstelling}
\label{sec:probleemstelling}

Dankzij deze scriptie krijgt men een duidelijk visueel zicht over hoe de momentele stand van zaken is op vlak van IPv6 en de adoptie ervan over heel de wereld. Deze zou een motivatie kunnen opleveren voor andere bedrijven de stap te laten nemen naar een IPv6 of een IPv4/IPv6 netwerk structuur. Sinds de lancering van IPv6 is er al een tijd verstreken. Daarom zal deze proef een update geven van waar de huidige situatie zich bevindt.

\section{Onderzoeksvraag}
\label{sec:onderzoeksvraag}

Enkele onderzoeksvragen die gesteld kunnen worden is hoe IPv4 en IPv6 zich met elkaar onderschijden. Wat de huidige situatie is van IPv6 op globaal niveau. Hoe goed België scoort op de adoptie van IPv6. Hoe Belgische bedrijven zich hierop gaan aanpassen. Hoe het komt dat België goed of slecht scoort en wat de bevindingen zijn van bedrijven. Alsook waarom de overschakeling niet zo vlot aan het verlopen is.

\section{Onderzoeksdoelstelling}
\label{sec:onderzoeksdoelstelling}

Om deze scriptie tot een succes te brengen is het noodzakelijk genoeg data en informatie te verkregen van bedrijven. Dankzij deze data is het mogelijk om de huidige stand van zaken voor te stellen en een beeld te scheppen hoe goed de adoptie van IPv6 aan het verlopen is.

\section{Opzet van deze bachelorproef}
\label{sec:opzet-bachelorproef}

% Het is gebruikelijk aan het einde van de inleiding een overzicht te
% geven van de opbouw van de rest van de tekst. Deze sectie bevat al een aanzet
% die je kan aanvullen/aanpassen in functie van je eigen tekst.

De rest van deze bachelorproef is als volgt opgebouwd:

In Hoofdstuk~\ref{ch:methodologie} wordt de methodologie toegelicht en worden de gebruikte onderzoekstechnieken besproken om een antwoord te kunnen formuleren op de onderzoeksvragen.

% TODO: Vul hier aan voor je eigen hoofstukken, één of twee zinnen per hoofdstuk

In Hoofdstuk~\ref{ch:h1}, in het eerste hoofdstuk zal er een inleiding gegeven worden over IPv6. Dankzij dit inleidend hoofdstuk zal er al een kennis verschaft worden die interessant kan zijn voor de rest van de scriptie. Alsook worden hierin de basiselementen uitgelegd over IPv6.

In Hoofdstuk~\ref{ch:h2}, het tweede hoofdstuk bevat een gedetailleerde vergelijking tussen IPv4 en IPv6. In dit hoofdstuk zal vooral de nadruk gelegd worden op de headers van beide protocollen.

In Hoofdstuk~\ref{ch:h3}, nu IPv6 steeds populairder wordt is het ook nodig om communicatie te leggen tussen IPv4 clients en IPv6 clients. In dit hoofdstuk zullen dus enkele transitie en tunnel technieken uitgelegd worden.

In Hoofdstuk~\ref{ch:h4}, het is nu al zeker dat er een einde zal komen aan IPv4. Hierin zal verder uitgelegd worden hoe de momentele stand van zaken is voor IPv4. Hoe lang er nog te resten valt voor IPv4 en hoe de komende maanden eruit zullen zien.

In Hoofdstuk~\ref{ch:h5}, hierin zal er duidelijk de huidige adoptie grafisch voorgesteld worden op globaal niveau. Aan de hand van verschillende data en grafieken is het zeer duidelijk hierover een zicht te verkrijgen.

In Hoofdstuk~\ref{ch:h6}, in dit hoofdstuk zal er dieper gegaan worden in de adoptie op de Belgische markt ne hoe sommige bedrijven ervoor staan. Dankzij opgestelde vragen krijgt men een voorbeeld hoe de Belgische bedrijven erover denken en ervoor staan. 

In Hoofdstuk~\ref{ch:conclusie}, tenslotte, wordt de conclusie gegeven en een antwoord geformuleerd op de onderzoeksvragen. Daarbij wordt ook een aanzet gegeven voor toekomstig onderzoek binnen dit domein.

