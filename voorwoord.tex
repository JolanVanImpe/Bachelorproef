%%=============================================================================
%% Voorwoord
%%=============================================================================

\chapter*{Woord vooraf}
\label{ch:voorwoord}

%% TODO:
%% Het voorwoord is het enige deel van de bachelorproef waar je vanuit je
%% eigen standpunt (``ik-vorm'') mag schrijven. Je kan hier bv. motiveren
%% waarom jij het onderwerp wil bespreken.
%% Vergeet ook niet te bedanken wie je geholpen/gesteund/... heeft
Deze bachelorproef wordt beschouwd voor het afsluiten van mijn opleiding Toegepaste Informatica op HoGent. Hiermee wil ik aantonen dat ik op een zelfstandige basis een onderzoek kan voeren en afronden.

Ik heb gekozen om mijn onderwerp van mijn bachelorproef rond het nieuwere internet protocol namelijk internet protocol versie 6 te maken. Ik selecteerde dit onderwerp omdat er een grote mogelijkheid bestaat dat dit een bepaalde rol zal spelen in de toekomst van een netwerkbeheerder. Omdat dit een nieuw protocol is, heb ik hiermee zelf weinig ervaring en vond het een unieke kans om deze uitdaging aan te gaan en hierover een bachelorproef te schrijven. IPv6 is niet onbekend maar het is een protocol dat velen kennen maar weinigen volledig begrijpen, daarom wou ik onderzoeken hoe het met onze Belgische markt gesteld is en hoe deze hierop anticipeert. Ook zal dit goed aansluiten met mijn studierichting wat het nog interessanter maakt voor mezelf.
Voor dit eindresultaat te kunnen leveren, stak ik er ontzettend veel tijd in. Dankzij enkele personen heb ik deze proef met succes tot zijn eind kunnen brengen en daarom wens ik hen even te bedanken voor alle steun en hulp die ik heb gekregen tijdens deze periode.

Eerst en vooral wil ik mijn promotor bedanken namelijk mevr. Karine Van Driessche, alsook mijn co-promotor Tim Rasschaert. Dankzij de hulp van hen, heb ik deze bachelorproef tot een einde kunnen volbrengen. Vervolgens zou ik mijn familie en vriendin willen bedanken die er steeds waren tijdens de moeilijke momenten tijdens deze periode. Zij gaven mij steeds weer moed om hier volop voor te gaan.
Verder hoop ik dat het lezen van deze proef u een deugddoend gevoel zal geven.
\begin{flushright}
\textit{Jolan Van Impe,} \\
\textit{Academiejaar 2017-2018}
\end{flushright}




