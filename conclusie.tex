%%=============================================================================
%% Conclusie
%%=============================================================================

\chapter{Conclusie}
\label{ch:conclusie}

%% TODO: Trek een duidelijke conclusie, in de vorm van een antwoord op de
%% onderzoeksvra(a)g(en). Wat was jouw bijdrage aan het onderzoeksdomein en
%% hoe biedt dit meerwaarde aan het vakgebied/doelgroep? Reflecteer kritisch
%% over het resultaat. Had je deze uitkomst verwacht? Zijn er zaken die nog
%% niet duidelijk zijn? Heeft het onderzoek geleid tot nieuwe vragen die
%% uitnodigen tot verder onderzoek?

%%\lipsum[76-80]

De conclusie die uit deze scriptie kan gehaald worden is dat men zowel op globaal als op Belgische vlak zeer sterk bezig is met de adoptie van IPv6. De meeste bedrijven, vooral ISP’s, zijn er zich van bewust dat het werken met IPv6 een vereiste zal worden. Ook de afgelopen jaren was er een toenemende groei van het gebruik hiervan en zal de komende maanden en jaren alleen maar verbeteren. Hiervoor is het ook zeker belangrijk om de groei op een jaarlijkse basis goed bij te houden zodanig men dit als motivatie voor bedrijven kan gebruiken. Ook is België het beste land met het aantal gebruikers die IPv6 hanteren wat een totale verrassing is voor vele Belgen.

Alsook biedt deze scriptie enkele antwoorden op de onderzoeksvragen. Er is een duidelijk overzicht van hoe de adoptie en aanvaarding van IPv6 staat op globaal niveau. Volgens Google stond deze op 23.33\% op 26 mei 2018. Wat het toch al meer dan 1/5 van al het verkeer maakt. Na 7 jaar sinds de eerste IPv6 wereld dag kan men dit zeker niet slecht noemen. Als er nog een duidelijker zicht wilt gegeven worden is het nodig om per land zijn evolutie te gaan bekijken.

Om enkele antwoorden te bieden op de onderzoeksvragen omtrent de evolutie op Belgisch niveau, dan kunnen we zeggen dat België het beter doet dan verwacht. Zoals eerder vernoemd, scoort België het beste op vlak van IPv6 gebruikers over heel de wereld wat het zeer interessant maakt. Om hierover een duidelijke verklaring te geven is moeilijk maar de gegeven veronderstellingen van Eric Vyncke kunnen hier zeker een invloed op hebben. Er kan ook afgeleid worden uit de enquêtes dat er toch al aan IPv6 gedacht wordt en dat hiermee al effectief in een werkomgeving mee geëxperimenteerd wordt. Alsook kan er geconcludeerd worden dat er nog veel groeipotentieel is naargelang het gebruik maken van IPv6 en dan niet zo gericht op de implementatie. Enkele hoofdredenen waarom bij de meeste bedrijven nog niet een implementatie is doorgekomen is vooral het gebrek aan kennis van het protocol. De noodzaak is dat bedrijven extra investeringen moeten doen omtrent trainingen, cursussen, certificatie en seminaries. Er kan ook gezegd worden dat de overgang van IPv4 naar IPv6 niet op één dag kan gebeuren maar dat dit vooral veel tijd, werk en planning in beslag neemt. Er wordt dus ook een IPv4/IPv6 verkozen in plaats van een IPv6 only netwerk. Wat vooral te maken kan hebben met de comptabiliteit van de apparaten in de infrastuur als met de kennis die engineers momenteel hebben over het protocol.

Er wacht IPv6 een veelbelovende toekomst te wachten waaraan bedrijven zich meer en meer bewust van beginnen te worden dat het hoe dan ook een vereiste is om hiermee in aanraking te komen. 
